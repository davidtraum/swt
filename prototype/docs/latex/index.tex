Hier ist eine kleine Anleitung wie man das Projekt auf seinem eigenen Rechner synchronisiert\+:


\begin{DoxyEnumerate}
\item git installieren
\item $>$$>$ git clone \href{https://github.com/davidtraum/swt/}{\texttt{ https\+://github.\+com/davidtraum/swt/}}
\item $>$$>$ cd swt
\end{DoxyEnumerate}

Wenn man was geändert hat\+:

(0. Ins Basisverzeichnis vom Projekt gehen)
\begin{DoxyEnumerate}
\item $>$$>$ git add $\ast$
\end{DoxyEnumerate}
\begin{DoxyEnumerate}
\item $>$$>$ git commit -\/m \char`\"{}\+Kurze Nachricht was man gemacht hat\char`\"{}
\item $>$$>$ git push origin master (Oder eigenen Branch angeben)
\end{DoxyEnumerate}

\#\+Changelog

\tabulinesep=1mm
\begin{longtabu}spread 0pt [c]{*{2}{|X[-1]}|}
\hline
\PBS\centering \cellcolor{\tableheadbgcolor}\textbf{ Datum  }&\PBS\centering \cellcolor{\tableheadbgcolor}\textbf{ Funktion   }\\\cline{1-2}
\endfirsthead
\hline
\endfoot
\hline
\PBS\centering \cellcolor{\tableheadbgcolor}\textbf{ Datum  }&\PBS\centering \cellcolor{\tableheadbgcolor}\textbf{ Funktion   }\\\cline{1-2}
\endhead
28.\+10.  &Start Changelog   \\\cline{1-2}
28.\+10.  &Animation beim Klick auf Städte   \\\cline{1-2}
28.\+10.  &Übersichtskarte mit Taste O   \\\cline{1-2}
\end{longtabu}
